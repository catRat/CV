\section{项目}
在就职于深圳鸿安货运代理有限公司期间,我被指派完成一项更新合约价格的工作。工作的材料是一份电子表格,表格出自我同事之手,我需要对它进行一系列操作,制作成另一个,被系统认可,可以上传至公司的某个系统电子表格;一系列操作的系列,包含了筛选、复制和粘贴等。总之,新表格包含了一些需求:只需要感兴趣的数据;增加新的变量(表中的一列)以标识着份表格;以某些条件修改观测(表中的行,我还记得公司的一个要求,在上传表格前,如果表格中某家船东合约价格起始日期如果比更新价格的日期晚,那么要把该船东合约价格起始日期改为更新系统的合约价格的日期)。这样的工作通常需要打开一个已存在的电子表格A,再新建一个空白的电子表格B,手工的从表格A中筛选出需要的数据,复制粘贴到电子表格B中。在公司里工作了几个月(我已经记不清是几个月)之后,我尝试使用R脚本语言编写脚本,然后使用window控制台批处理命令运行该脚本来完成一直以来只能通过我勤奋的双手在键盘上敲击和握着鼠来回移动才能完成的工作。经过多次的修改之后,脚本终于好到足以正确地完成它肩负的任务。当然,这份脚本是渺小的,它在我需要的时候被用到,其它时候则安静的躺在硬盘里。R语言是一门开源的脚本语言,它是一种在统计领域应用广泛的脚本语言;我的有关R语言知识,一部分来自老师的教学,一部分来自自学。后来,它被我保存在\url{https://github.com/catRat/my-first-work}的update目录下。
